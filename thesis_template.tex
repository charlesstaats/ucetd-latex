% ---- ETD Document Class and Useful Packages ---- %
% Note: adding the [safe] option before {ucetd} will produce slightly inferior 
% spacing, but may fix certain errors.
\documentclass{ucetd}
\usepackage{natbib}
\usepackage{amsmath}
\usepackage{amssymb}
\usepackage{amsthm}

% NOTE: The following package is included for dummy text only.
% It should be omitted from your final draft.
\usepackage{lipsum}

%% Set the bibliography style here; look up information on natbib for more info.
%% (Comment this line out if you are using biblatex rather than the older, but more standard, bibtex.)
\AtBeginDocument{\bibliographystyle{plainnat}}

%% Use these commands to set biographic information for the title page:
\title{Thesis Title}
\author{Thesis Author}
\departmentname{Thesis Department}
\divisionname{Thesis Division} % old version \division causes error in safe mode
\degreename{Type of Degree} % old version \degree causes error in safe mode
\date{Graduation Date}

%% Use these commands to set a dedication and epigraph text
\dedication{Dedication Text}
\epigraph{Epigraph Text}

%% Specify any supplementary files that should be added to the end of the table of contents:
\supplementaryfile{Video Clips from President Zimmerman's Inauguration}
\supplementaryfile[\texttt{temperatures.dat} (supplementary file)]{Daily Temperature Records at Many Locations for the Last Fifty Years}

\begin{document}
%% Basic setup commands
% If you don't want a title page comment out the next line and uncomment the line after it:
\maketitle
%\omittitle

% These lines can be commented out to disable the copyright/dedication/epigraph pages
\makecopyright
\makededication
\makeepigraph


%% Make the various tables of contents
\tableofcontents
\listoffigures
\listoftables

\acknowledgments
% Enter Acknowledgements here

\abstract
% Enter Abstract here

\mainmatter
% Main body of text follows

\chapter{Introduction}
% Introductory stuff

Here's a list of things:
% To double-space individual items (not recommended), use the environments ending in DS.
%\begin{itemizeDS}
\begin{itemize}
\item The first item
\item The second item
\item \lipsum*[4]

\lipsum*[5]
%\end{itemizeDS}
\end{itemize}
% Similarly, you can use enumerate to get a numbered list, and enumerateDS to get a numbered 
% list in which individual items are double-spaced (again, this is permitted but not recommended).

\chapter{A Chapter}
\section{Introduction}
% Intro to chapter one

% Format a LaTeX bibliography
% Specify the filename(s) of your .bib file(s), separated by commas
\makebibliography{filename1,filename2}
%% IF USING BIBLATEX: Comment out (or delete) the line above. Specify the filename(s) 
%% using \addbibresource in the preamble as usual, then construct the bibliography 
%% by un-commenting the following line:
%\makebibliography

% Figures and tables, if you decide to leave them to the end
%\input{figure}
%\input{table}

\end{document}
